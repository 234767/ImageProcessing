\usepackage{tabularx}
\usepackage[a4paper,margin=2.5cm, bottom=3.5cm]{geometry}
\usepackage{fancyhdr}
\usepackage{listings}
\usepackage{booktabs}
\usepackage{float}
\usepackage{subcaption}
\usepackage{graphicx}
\usepackage{amsmath}
\usepackage{amssymb}
\usepackage{amsthm}
\usepackage{array}
\usepackage[table]{xcolor}
\usepackage{pgfplots}
\usepackage{pgfplotstable}
\usepackage{multirow}
\usepackage{tikz}
\usepackage[hidelinks]{hyperref}
\usepackage{titling}
\pgfplotsset{compat=1.17}

\theoremstyle{definition}
\newtheorem*{example}{Example}
\setlength{\headheight}{40pt}
\setlength{\parindent}{0pt}
\setlength{\parskip}{1ex}
\renewcommand{\headrulewidth}{0pt}

\newcommand{\subfiguresize}{.3\textwidth}
\DeclareMathOperator*{\median}{median}

\newcommand{\biggerforall}{\mbox{\Large $\mathsurround0pt\forall$}} 
\newcommand{\bigforall}{\mbox{\large $\mathsurround0pt\forall$}} 
\newcommand{\biggerexists}{\mbox{\Large $\mathsurround0pt\exists$}} 
\newcommand{\bigexists}{\mbox{\large $\mathsurround0pt\exists$}} 

\lstset {
    captionpos = b,
    basicstyle = \small\ttfamily,
    keywordstyle = \color{blue},
    commentstyle = \color{black!30},
    comment = [l]{//},
    morecomment = [s]{/*}{*/},
    identifierstyle=,
    keywords = {
        let,
        mut,
        for,
        in,
        if,
        else,
        continue,
        break,
        pub,
        struct,
        impl,
        type,
        Self,
        u8,u16,u32,u64,
        i8,i16,i32,i64,
        f32,f64,
        usize,
    }
}

\pagestyle{fancy}
\fancyhead{}
\fancyhead[L]{
    \renewcommand{\arraystretch}{1.5}
    \begin{tabularx}{\textwidth}{|X|X|}
        \hline
        \large \bf Image processing & \normalsize \thetitle \\
        \hline
    \end{tabularx}
}
\fancyfoot[C]{\thepage}

\renewcommand{\maketitle}{
    \thispagestyle{plain}
    \renewcommand{\arraystretch}{2}
    \begin{flushleft}
        \begin{tabularx}{0.95\textwidth}{|X|X|}
            \hline
            \bf \large Image Processing                   & \bf \large \thetitle                           \\ \hline
            \multicolumn{2}{|l|}{
                \textbf{Task variant:} Group 1
            }                                                                                               \\ \hline
            \textbf{Day and time:} Mon, 14:00             & \textbf{Full name:} \textsc{Jakub Pawlak}       \\
            \textbf{Academic year:} {2022/23} & \textbf{Full name:} \textsc{Magdalena Paku\l a} \\
            \hline
        \end{tabularx}
    \end{flushleft}
    \vspace{1em}
    \renewcommand{\arraystretch}{1}
}