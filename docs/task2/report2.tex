\documentclass[12pt]{article}

\usepackage{tabularx}
\usepackage[a4paper,margin=2.5cm, bottom=4cm]{geometry}
\usepackage{fancyhdr}
\usepackage{listings}
\usepackage{booktabs}
\usepackage{float}
\usepackage{subcaption}
\usepackage{graphicx}
\usepackage{amsmath}
\usepackage{amssymb}
\usepackage{amsthm}
\usepackage{array}
\usepackage[table]{xcolor}
\usepackage{pgfplots}
\usepackage{pgfplotstable}
\usepackage{tikz}
\pgfplotsset{compat=1.17}

\theoremstyle{definition}
\newtheorem*{example}{Example}

\graphicspath{{../img/}}

\setlength{\headheight}{40pt}
\setlength{\parindent}{0pt}
\setlength{\parskip}{1ex}
\renewcommand{\headrulewidth}{0pt}

\newcommand{\subfiguresize}{.3\textwidth}
\DeclareMathOperator*{\median}{median}

\lstset {
    basicstyle = \small\ttfamily,
    keywordstyle = \color{blue},
    commentstyle = \color{black!30},
    comment = [l]{//},
    morecomment = [s]{/*}{*/},
    identifierstyle=,
    keywords = {
        let,
        mut,
        for,
        in,
        if,
        else,
        pub,
        struct,
        impl,
        type,
        Self,
        u8,u16,u32,u64,
        i8,i16,i32,i64,
        f32,f64,
    }
}

\begin{document}

\pagestyle{fancy}
\fancyhead{}
\fancyhead[L]{
    \renewcommand{\arraystretch}{1.5}
    \begin{tabularx}{\textwidth}{|X|X|}
        \hline
        \large \bf Image processing & \normalsize Task No. 1 \\
        \hline
    \end{tabularx}
}
\fancyfoot[C]{\thepage}

\thispagestyle{empty}
\renewcommand{\arraystretch}{2}
\begin{flushleft}
    \begin{tabularx}{0.95\textwidth}{|X|X|}
        \hline
        \bf \large Image Processing                   & \bf \large Task No.~2                           \\ \hline
        \multicolumn{2}{|l|}{
            \textbf{Task variant:} Group 1
        }                                                                                               \\ \hline
        \textbf{Day and time:} Mon, 14:00             & \textbf{Full name:} \textsc{Magdalena Paku\l a}       \\
        \textbf{Academic year: 3\textsuperscript{rd}} & \textbf{Full name:} \textsc{Jakub Pawlak} \\
        \hline
    \end{tabularx}
\end{flushleft}
\vspace{1em}
\renewcommand{\arraystretch}{1}

\section{Description of the implementation of the histogram-based image enhancement method}

\section{Image analysis on the basis of the histogram}

\section{Description of the linear filter implementation (general formulation)}

\section{Description of the linear filter implementation (with optimization)}

\section{Analysis of the filtering results (linear filter)}

\section{Description of the non-linear filter implementation }

\section{Analysis of the filtering results (non-linear filter)}

\section{Description of other changes which took place in the application}



\vfill
\section*{Teacher's remarks}
\begin{tabularx}{\textwidth}{|X|}
    \hline
    \vspace{7cm}
    \phantom{.} \\
    \hline
\end{tabularx}

\end{document}