\documentclass[12pt]{article}

\usepackage{tabularx}
\usepackage[a4paper,margin=2.5cm, bottom=3.5cm]{geometry}
\usepackage{fancyhdr}
\usepackage{listings}
\usepackage{booktabs}
\usepackage{float}
\usepackage{subcaption}
\usepackage{graphicx}
\usepackage{amsmath}
\usepackage{amssymb}
\usepackage{amsthm}
\usepackage{array}
\usepackage[table]{xcolor}
\usepackage{pgfplots}
\usepackage{pgfplotstable}
\usepackage{multirow}
\usepackage{tikz}
\usepackage[hidelinks]{hyperref}
\usepackage{titling}
\pgfplotsset{compat=1.17}

\theoremstyle{definition}
\newtheorem*{example}{Example}
\setlength{\headheight}{40pt}
\setlength{\parindent}{0pt}
\setlength{\parskip}{1ex}
\renewcommand{\headrulewidth}{0pt}

\newcommand{\subfiguresize}{.3\textwidth}
\DeclareMathOperator*{\median}{median}

\DeclareMathOperator*{\biggerforall}{\mbox{\Large $\mathsurround0pt\forall$}} 
\DeclareMathOperator*{\bigforall}{\mbox{\large $\mathsurround0pt\forall$}} 
\DeclareMathOperator*{\biggerexists}{\mbox{\Large $\mathsurround0pt\exists$}} 
\DeclareMathOperator*{\bigexists}{\mbox{\large $\mathsurround0pt\exists$}} 

\lstset {
    captionpos = b,
    basicstyle = \small\ttfamily,
    keywordstyle = \color{blue},
    commentstyle = \color{black!30},
    comment = [l]{//},
    morecomment = [s]{/*}{*/},
    identifierstyle=,
    keywords = {
        let,
        mut,
        for,
        in,
        if,
        else,
        continue,
        break,
        pub,
        struct,
        impl,
        type,
        self,
        Self,
        as,
        u8,u16,u32,u64,
        i8,i16,i32,i64,
        f32,f64,
        usize,
    }
}

\pagestyle{fancy}
\fancyhead{}
\fancyhead[L]{
    \renewcommand{\arraystretch}{1.5}
    \begin{tabularx}{\textwidth}{|X|X|}
        \hline
        \large \bf Image processing & \normalsize \thetitle \\
        \hline
    \end{tabularx}
}
\fancyfoot[C]{\thepage}

\renewcommand{\maketitle}{
    \thispagestyle{plain}
    \renewcommand{\arraystretch}{2}
    \vspace*{-7em}
    \begin{flushleft}
        \begin{tabularx}{0.95\textwidth}{|X|X|}
            \hline
            \bf \large Image Processing                   & \bf \large \thetitle                           \\ \hline
            \multicolumn{2}{|l|}{
                \textbf{Task variant:} Group 1
            }                                                                                               \\ \hline
            \textbf{Day and time:} Mon, 14:00             & \textbf{Full name:} \textsc{Jakub Pawlak}       \\
            \textbf{Academic year:} {2022/23} & \textbf{Full name:} \textsc{Magdalena Paku\l a} \\
            \hline
        \end{tabularx}
    \end{flushleft}
    \vspace{1em}
    \renewcommand{\arraystretch}{1}
}
\graphicspath{{../img_task2/}}

\title{Task No.~2}

\begin{document}
\maketitle

\section{Description of the implementation of \texorpdfstring{\\ }{} histogram-based image enhancement method}
Histogram based image enhancements work by assigning new values to the luminosity levels, to match the histogram to some function.
The old luminosities will be denoted as $f$, and the new ones, as $g$.
The histogram of the image is denoted as $H$, and $H(f)$ means the number of pixels with luminosity $f$.
Finally, $N$ will mean the total number of pixels in the image.

\subsection{Mathematical description}

The equation for calculating the new luminosity values were given as:
\begin{equation}
    g(f) = g_{min} +\sqrt{
        2 \alpha^2 \cdot \ln\left[
            \left(
            \frac{1}{N} \sum\limits_{k=0}^f H(k)
            \right)^{-1}
            \right]
    }
    \label{eq:rayleigh-initial}
\end{equation}

We immediately noticed, that the numer under the logarithm is a partial sum of the normalized histogram values for luminosities smaller or equal $f$.
We can refactor the equation, by introducing a partial sum function $ps(f)$:
\begin{equation}
    ps(f) = \sum\limits_{k=0}^f H(f)
\end{equation}

Equation (\ref{eq:rayleigh-initial}) then takes form:
\begin{equation}
    g(f) = g_{min} + \sqrt{
        2 \alpha^2 \cdot \ln
        \left(
        \frac{N}{ps(f)}
        \right)}
\end{equation}

Note that,
\(
\bigforall_{f \in \langle f_{min},f_{max}\rangle} : 0 < ps(f) \leq N
\), and $ps(f)$ is monotinically non-decreasing.
That means that $\frac{N}{ps(f)}$ is monotinically non-increasing, and as follows, $g(f)$ is also non-increasing.

However, this is not what we want from image transformation, as it would produce a negative image.
Therefore, we flip the function horizontally, by changing the sign of $ps(f)$, and offseting it by $N$, to keep the domain the same.

\begin{equation}
    g(f) = g_{min} + \sqrt{
        2 \alpha^2 \cdot \ln
        \left(
        \frac{N}{N - ps(f) + 1}
        \right)}
    \label{eq:rayleigh-corrected}
\end{equation}
Now, as $f$ increases, $g(f)$ will increase as well.

However, the value of $\alpha$ is still unknown.
We know, that the function $g(f)$ should map $f$ to $g$ in such a way, that
$f_{min} \mapsto g_{min}$ and $f_{max} \mapsto g_{max}$.

For the first case:
\(
\lim\limits_{f \to f_{min}} ps(f) = 1
\),which means that
\(
\lim\limits_{f \to f_{min}} g(f) = g_{min}
\) regardless of $\alpha$.

The second case is more interesting:
\begin{align*}
    ps(f_{max}) & = N                \\[1ex]
    g(f_{max})  & = g_{min} + \sqrt{
        2 \alpha^2 \cdot \ln(N)
    }
\end{align*}
Substituting $g(f_{max}) = g_{max}$,
\begin{align}
    g_{max} & = g_{min} + \sqrt{
        2 \alpha^2 \cdot \ln(N)
    }\nonumber                                                                     \\[1ex]
    g_{max} & = g_{min} + \alpha \cdot \sqrt{2\ln(N)} \nonumber                    \\[1ex]
    \alpha  & = \frac{g_{max} - g_{min}}{\sqrt{2\ln(N)}} \label{eq:rayleigh-alpha}
\end{align}

\subsection{Implementation}

We start, by calculating the partial sums of the histogram for each channel:
\begin{lstlisting}
let partial_sums: [[u32; 256]; 3] = {
        let histogram = Histogram::new(image);
        histogram
            .into_iter()
            .map(|h| {
                h.iter()
                    .scan(0u32, |sum, value| {
                        *sum += value;
                        Some(*sum)
                    })
                    .collect()
            })
            .collect::<Vec<[u32; 256]>>()
            .try_into()
            .unwrap()
    };
\end{lstlisting}

\pagebreak[2]
Then, we calculate the image size, and the value of $\alpha$, according to eq. (\ref{eq:rayleigh-alpha})
\begin{lstlisting}
let image_size = image.width() * image.height();
let alpha = (self.gmax - self.gmin) as f64 
    / f64::sqrt(2.0 * f64::ln(image_size as f64));
\end{lstlisting}

\pagebreak[2]
Then, we create the brighness lookup table, according to eq. (\ref{eq:rayleigh-corrected}):
\begin{lstlisting}[basicstyle = \footnotesize\ttfamily]
let mut brightness_lookup = [[0u8; 256]; 3];
for channel in 0..3 {
    for i in 0..256 {
        let partial_sum = partial_sums[channel][i];
        if partial_sum == 0 {
            continue;
        }
        let log_base = 
            image_size as f64 / (image_size - partial_sum + 1) as f64;
        let root_base = 
            2.0 * alpha * alpha * f64::ln(log_base);
        brightness_lookup[channel][i] = 
            self.gmin
            + f64::clamp(
                f64::sqrt(root_base), 
                0.0, 
                (self.gmax - self.gmin) as f64) as u8;
        if partial_sums[channel][i] == image_size {
            break;
        }
    }
}
\end{lstlisting}

Note 2 important things --- the \lstinline{continue} and \lstinline{break} statements inside the loop.
We are interested in filling the lookup table only for values of $f \in \langle f_{min}, f_{max} \rangle$.
Therefore, we check if $ps(f) = 0$. If it is, then $f < f_{min}$, and we need not perform the calculation, so we continue looping
until we encounter some $ps(f) > 0$.

Similarly, we do not need to perform calculations for $f > f_{max}$. We know that $ps(f) = N \Leftrightarrow f \geq f_{max}$,
so if we reach that point, we do not need to perform any calculations for further values of $f$, and can safely break out of the loop.

This is a small optimization compensating for the fact that we cannot create a smaller lookup table, adjusted to the size of $\langle f_{min}, f_{max}\rangle$, because these values are unknown at compile time.

\pagebreak[2]
Finally, we set the new luminosity levels in the image according to the created lookup table:
\begin{lstlisting}
for pixel in image.pixels_mut() {
    for channel in 0..3 {
        let luminosity = pixel[channel];
        let new_luminosity = 
            brightness_lookup[channel][luminosity as usize];
        pixel[channel] = new_luminosity
    }
}
\end{lstlisting}


\section{Image analysis on the basis of the histogram}

\section{Description of the linear filter implementation (general formulation)}

\section{Description of the linear filter implementation (with optimization)}

\section{Analysis of the filtering results (linear filter)}

\section{Description of the non-linear filter implementation }

\section{Analysis of the filtering results (non-linear filter)}

\section{Description of other changes which took place in the application}



\vfill
\section*{Teacher's remarks}
\begin{tabularx}{\textwidth}{|X|}
    \hline
    \vspace{7cm}
    \phantom{.} \\
    \hline
\end{tabularx}

\end{document}